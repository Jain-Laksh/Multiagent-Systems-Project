\section{Experimental Setup}

All experiments were conducted on the \texttt{CartPole-v1} environment from the Gymnasium library. The goal is to keep the pole upright for as long as possible, with a maximum episode duration of 500 steps. The episode terminates if the pole angle exceeds $\pm 12^\circ$ or the cart position exceeds $\pm 2.4$ units.

\subsection{Hyperparameter Configuration}
Table \ref{tab:hyperparams} summarizes the final hyperparameters selected after tuning. A distinct difference in configuration was necessary for the two families of algorithms.

\begin{table}[htbp]
\caption{Hyperparameter Settings}
\begin{center}
\begin{tabular}{|l|c|c|}
\hline
\textbf{Parameter} & \textbf{Value-Based} & \textbf{Policy-Based} \\
\hline
Learning Rate (LR) & $1 \times 10^{-4}$ & Actor: $3 \times 10^{-4}$ \\
 & & Critic: $1 \times 10^{-3}$ \\
\hline
Hidden Units & $64 \times 64$ & $128 \times 128$ \\
\hline
Gamma ($\gamma$) & 0.99 & 0.99 \\
\hline
Batch Size & 64 & N/A (Monte Carlo) \\
\hline
Memory Size & 10,000 & N/A \\
\hline
Soft Update ($\tau$) & 0.01 & N/A \\
\hline
Exploration ($\epsilon$) & $1.0 \to 0.001$ & Stochastic Policy \\
\hline
\end{tabular}
\label{tab:hyperparams}
\end{center}
\end{table}

\subsection{Stability and Catastrophic Forgetting}
A critical observation during our experimentation was the sensitivity of policy-based methods to hyperparameter selection compared to value-based methods. 

While DQN and SARSA showed robust convergence with standard hyperparameters, REINFORCE and Actor-Critic exhibited signs of \textit{catastrophic forgetting}—a phenomenon where the agent's performance dramatically collapses after a period of high rewards. Specifically, with higher learning rates or smaller network capacities (64 units), policy agents would solve the environment (reaching 500 reward) around episode 200-300, only to degrade to near-random performance by episode 400.

To mitigate this, we:
\begin{enumerate}
    \item Increased the network capacity to 128 hidden units.
    \item Tuned the learning rates, specifically using a lower rate for the Actor ($3 \times 10^{-4}$) compared to the Critic ($1 \times 10^{-3}$) to ensure policy updates did not outpace value estimation accuracy.
    \item Limited the training horizon for policy methods to the point of convergence to prevent late-stage divergence.
\end{enumerate}